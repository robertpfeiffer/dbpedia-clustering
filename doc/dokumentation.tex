%&program=xelatex
%&encoding=UTF-8 Unicode
\documentclass[a4paper,12pt]{article}

\usepackage[ngerman]{babel}
\usepackage{fontspec}

\title{DBPedia-Clustering mit dem K-Means-Algorithmus}

\author{Robert Pfeiffer \and Tobias Schmidt}
\setromanfont[Mapping=tex-text,Alternate=1,Ligatures={Common,Diphthong}]{Palatino} 
%%Die Schrift aus den DBS-Übungsblättern

\begin{document}
\maketitle
\section{Einleitung}
Problemstellung hier wiederholen

Clustering erklären

DBPedia erklären

\section{Der K-means-Algorithmus}

Clusterzentren auswählen

subjekte dem nächsten zentrum zuordnen
daraus neue cluster bilden
den mittelwert der cluster berechnen
daraus clusterzentren bilden
wiederholen
konvergenz

\subsection{freie Parameter}
anzahl der cluster

vorauswahl der cluster (zufällig)

abstandsfunktion (jaccard/euklid)

mittelwertsfunktion (arithmetisches mittel)

datenstruktur für zentren (8-bit festkomma-array)

datenstruktur für abstand (float)

anzahl der iterationen/abbruchbedingung

eingabeformat (matrix oder listen)

% implementierungsdetails ab hier
\section{Abstandsmaße}
\subsection{euklidischer abstand}
\subsection{Der Jaccard-Abstand}
\subsection{Fuzzy-Mengen}

\section{Verteilung auf hadoop}
\subsection{Karthesisches Produkt}
-langsam

-viel I/O

-mehr zyklen

-inputformat

\subsection{Caching von Zentren}
-machen wir

\section{Eingabeformat}

\section{Performance/Laufzeitkomplexität}

\end{document}