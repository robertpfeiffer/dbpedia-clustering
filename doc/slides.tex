\documentclass[xelatex]{beamer}


\usepackage{listings}
\usepackage{pgf}
\usepackage{euler}

\usepackage{xunicode,fontspec,xltxtra}

\setromanfont[Alternate=1,Ligatures={Common}]{Hoefler Text}
\setsansfont{Futura}%{Linux Biolinum O}
\setmonofont{Inconsolata}

\mode<presentation>
{
   \setbeamercovered{highly dynamic}
   
    \usetheme{default}
    %\usefonttheme{serif}
    \setbeamertemplate{navigation symbols}{} 
}
 

\titlegraphic{\includegraphics[scale=0.05]{./hpi_logo_gr.jpg}}
\title{K-means Clustering of DBpedia-Subjects}
\author[Robert Pfeiffer and Tobias Schmidt] {Robert Pfeiffer\\Tobias Schmidt} 
\date{\today} 
	
\begin{document}


%\maketitle
%\tableofcontents
%\section{bla}


\begin{frame}{Map- und Reduce- Schritte}
  Map

  Distanz zwischen einem Subjekt s und einem Clusterzentrum c
  $$ dist(c,s) = \sum_{i=1}^{n} | s_i \cdot 255 - c_i |^2 $$
  
  \pause
  Reduce
  
  n Subjekte im Cluster C werden gemittelt
  $$ mean(C) = \left\lfloor \frac{255}{n} \sum_{\vec s \in C}  \vec s \right\rfloor $$
  \pause
  Clusterzentren sind Vektoren von Bytes
  
\end{frame}

\end{document}