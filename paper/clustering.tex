\documentclass[a4paper]{llncs}
\usepackage{llncsdoc}

%% Deutsche Anpassungen %%%%%%%%%%%%%%%%%%%%%%%%%%%%%%%%%%%%%
\usepackage[ngerman]{babel}
\usepackage[T1]{fontenc}
\usepackage[utf8]{inputenc}

% Paket für Graphiken
\usepackage[pdftex]{graphicx}

%\Paket für Hyperlinks
\usepackage{hyperref}

\title{Clustering of DBPedia Subjects}
\author{Robert Pfeiffer, Tobias Schmidt}
\institute{Hasso-Plattner-Institut für Softwaresystemtechnik, Potsdam}
\begin{document}
\maketitle
\begin{abstract}
TODO
\end{abstract}

\section{Einleitung}
- Was ist Map/Reduce, was ist Hadoop\\
- Für welche Aufgaben eignet sich dieser Ansatz

\section{Clustering}

\section{Implementierung}

\subsection{k-means Algorithmus}
- 2 Phasen. Welche Daten werden von Mapper an Reducer und umgekehrt übergeben\\
- Vorteile / Nachteile gegenüber anderen Ansätzen\\
- <überarbeitete Grafik ?>

\subsection{Distributed Cache}
- Problem: Verteilung der Centroids\\
- Lösung mittels Distributed Cache\\
- Vorteile: in Hadoop, synchrone Datenhaltung, wenig Overhead für die restlichen berechnungen\\
- Nachteile: bricht das Map/Reduce Konzept\\
- andere Möglichkeit: kartesisches Produkt

\subsection{Sequence Files}
- Problem: Daten sind große Matrizen\\
- Vorteile:\\
    - Hadoop Format, dadurch gibt es bereits ein Interface und Hadoop kann den Input splitten\\
    - komprimiert\\
- Nachteile:\\
    - kodiertes Format, daher für Menschen nicht lesbar\\
    - keine Information über die Gesamtgröße

\subsection{Festkommazahlen}
- Centroids werden durch Vektor über Festkommazahlen mit der Größe eines Bytes dargestellt\\
- Vorteile: klein Datengröße\\
- Nachteile: evt. zu grobe Abstufungen, man muss immer an die Umrechnung denken, unflexibel, schlecht austauschbar

\section{Evaluierung}
Eingabegröße
Anzahl der Cluster
Abbruch

\section{Fazit}
TODO

\begin{thebibliography}{--------}

\bibitem[KI2008]{KI2008}
	Toby Segaran.
  {\em Kollektive Intelligenz}.
   O'Reilly, 2008
   
\end{thebibliography}

\end{document}
